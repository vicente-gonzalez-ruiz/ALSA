\title{Advanced Linux Sound Architecture (ALSA)}

\maketitle

\begin{figure}
  \begin{center}
    \begin{tabular}{|c|c|}
      \hline
      \multicolumn{2}{c}{\href{https://en.wikipedia.org/wiki/List\_of\_Linux\_audio\_software}{Linux Audio Applications}} \\
      \hline
      \href{https://jackaudio.org/}{JACK} & \href{https://www.freedesktop.org/wiki/Software/PulseAudio/}{Pulse Audio} \\
      \hline
      \multicolumn{2}{c}{\href{https://www.alsa-project.org}{ALSA}} \\
      \hline
      \multicolumn{2}{c}{\href{https://www.alsa-project.org/wiki/Matrix:Main}{Sound hardware}} \\
      \hline
    \end{tabular}
  \end{center}
  \caption{The Linux Audio (problably partial) Stack.}
  \label{fig:stack}
\end{figure}

ALSA is a software
\href{https://docs.kernel.org/sound/kernel-api/index.html}{framework}
and part of the Linux kernel (see Fig.~\ref{fig:stack}), as a set of
\href{https://wiki.archlinux.org/title/Kernel_module}{modules}, that
provides an application programming interface (API), in
\href{https://en.wikipedia.org/wiki/C_(programming_language)}{C}, for
sound device drivers~\cite{phillips2005user}. For example, to show the
modules related with ALSA currently loaded, run:
\begin{lstlisting}[language=Bash]
lsmod | grep '^snd' | column -t
\end{lstlisting}

Let's see some of the most basic tools shipped with ALSA.

\section{\texttt{amixer} and \texttt{alsamixer}}
\texttt{\href{https://linux.die.net/man/1/amixer}{amixer}} (through
the command line) and
\texttt{\href{https://en.wikipedia.org/wiki/Alsamixer}{alsamixer}}
(using also the command line but being ncurses-based) can be used to
control the gain of the audio inputs and outputs. For example, to run
\texttt{alsamixer}, in your terminal write:
\begin{lstlisting}[language=Bash]
alsamixer
\end{lstlisting}

\section{\texttt{arecord}}
\texttt{\href{https://linux.die.net/man/1/arecord}{arecord}} runs on
ALSA to capture
\href{https://en.wikipedia.org/wiki/Pulse-code_modulation}{PCM (Pulse
  Code Modulation)} (by default, using the
\href{https://en.wikipedia.org/wiki/WAV}{WAV} audio format) audio
data, and provides information about the hardware, and also, about the
audio servers. Some examples of use are:

\begin{itemize}

\item List the physical audio
  \href{https://en.wikipedia.org/wiki/Analog-to-digital_converter}{ADC}(s):
\begin{lstlisting}[language=Bash]
arecord -l      
\end{lstlisting}

\item Show information about the ADC(s) capabilities (if
  \href{https://www.freedesktop.org/wiki/Software/PulseAudio/}{PulseAudio}
  is not running, remove \texttt{pasuspender --}):
\begin{lstlisting}[language=Bash]
pasuspender -- arecord -D hw:0,0 --dump-hw-params
\end{lstlisting}

\item Idem, but using the default audio server (usually, PulseAudio):
\begin{lstlisting}[language=Bash]
arecord --dump-hw-params > /dev/null
\end{lstlisting}
Usually, the software capabilities are higher than the hardware ones.

\item List the virtual ADC(s)
  (\href{https://alsa.opensrc.org/Pcm-device}{PCM} outputs):
\begin{lstlisting}[language=Bash]
arecord -L
\end{lstlisting}

\item Record raw (without header) audio in
  \href{https://en.wikipedia.org/wiki/Compact_Disc_Digital_Audio#Audio_format}{CD
    audio format}:
\begin{lstlisting}[language=Bash]
arecord -t raw -f cd > signed_int16_bits__little_endian__44100_frames_per_second__2_samples_per_frame.raw
\end{lstlisting}

\end{itemize}

\section{\texttt{aplay}}
\texttt{\href{https://linux.die.net/man/1/aplay}{aplay}} is the
companion PCM player of \texttt{arecord}. The playing capabilities
used to be exactly the same than the provides by \texttt{arecord}.

\begin{itemize}

\item List the physical audio
  \href{https://en.wikipedia.org/wiki/Digital-to-analog_converter}{DAC}(s):
\begin{lstlisting}[language=Bash]
aplay -l      
\end{lstlisting}

\item Idem, but listing the virtual DACs (PCM outputs):
\begin{lstlisting}[language=Bash]
aplay -L
\end{lstlisting}

\item (Recording and) Playing directly through ALSA, using the physical DAC and DAC:
\begin{lstlisting}[language=Bash]
arecord -t raw -f cd | aplay -t raw -f cd
pasuspender -- arecord -D hw:0,0 -t raw -f cd | pasuspender -- aplay -D hw:0,0 -t raw -f cd
\end{lstlisting}

\item (Recording and) Playing throught the default sound server (usually PulseAudio), using the virtual DAC and DAC:
\begin{lstlisting}[language=Bash]
arecord -t raw -f cd | aplay -t raw -f cd
\end{lstlisting}
\end{itemize}

\section{\texttt{speaker-test}}
\texttt{\href{https://linux.die.net/man/1/speaker-test}{speaker-test}} allows to generate (by default) \href{https://en.wikipedia.org/wiki/Pink_noise}{pink noise} or a pure tone through the audio outputs. Examples:
\begin{lstlisting}[language=Bash]
speaker-test                       # Analog (copper or Bluetooth, if available) mono
speaker-test -c 2                  # Analog (copper of Bluethooh, if available) stereo
speaker-test -Dplug:spdif -c2      # Digital (SPDIF) stereo, if available
speaker-test -c 8                  # Surround (usually through HDMI) 7.1, if available
\end{lstlisting}

\section{Resources}

\bibliography{python,sound}
